\input{pack.tex}

\title{\LARGE \textbf{MiniRT} - Notes}
\author{\large Lucie Le Briquer}
\date{\today}

\begin{document}
\maketitle
\tableofcontents
\newpage
\section{Génération des rayons}
\newpage
\section{Rotation de la caméra}
\newpage
\section{Interaction rayon-sphère}
\newpage
\section{Interaction rayon-plan}
\newpage
\section{Interaction rayon-triangle}
\newpage
\section{Interaction rayon-carré}
\newpage
\section{Interaction rayon-cylindre}
\ni Soit $(o,X,Y,Z)$ le repère du cylindre $\Cl$. Le point $p\in\Cl$ ssi
$$x_p^2 + y_p^2 = r^2\etsp|z_p|\leq\frac h 2$$
\ni avec $(x_p,y_p,z_p)$ les coordonnées de $p$ dans le repère du
cylindre. Pour déterminer les vecteurs $(X,Y,Z)$ il suffit d'utiliser la même
construction de base vue précédemment avec comme axe de départ l'axe du cylindre.
\dd Le point $p$ est de la forme $o_R + \bt d_R$ où $o_R$ est l'origine du rayon et
$d_R$ sa direction. Ainsi :
$$x_p = \lng p - o, X\rng = \bt\scl{d_R, X} + \scl{o_r - o, X}$$
$$y_p = \lng p - o, Y\rng = \bt\scl{d_R, Y} + \scl{o_r - o, Y}$$
Donc,
\begin{align*}
	x_p^2 + y_p^2 = &\left(\scl{d_R,X}^2 + \scl{d_r,Y}^2\right) \bt^2\\
		&+2\left(\scl{d_R,X}\scl{o_R-o,X}+\scl{d_R,Y}\scl{o_R-o,Y}\right) \bt \\
		&+\left(\scl{o_R-o, X}^2 + \scl{o_R-o,Y}^2\right)
\end{align*}
\ni$\bt$ vérifie donc une équation du second degré $a\bt^2+b\bt+c=0$ avec :
$$\sys{a = \scl{d_R,X}^2 + \scl{d_r,Y}^2\\\\
b = 2\left(\scl{d_R,X}\scl{o_R-o,X}+\scl{d_R,Y}\scl{o_R-o,Y}\right) \\\\
c = \scl{o_R-o, X}^2 + \scl{o_R-o,Y}^2 - r^2
}$$
\ni Il ne reste plus qu'à vérifier que pour un des deux $\bt$ on obtient
un point $p$ vérifiant $|z_p|\leq\frac h 2$ sachant que :
$$z_p = \lng p - o, Z\rng = \bt\scl{d_R, Z} + \scl{o_r - o, Z}$$

\end{document}
